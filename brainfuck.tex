\subsection{History}

\blindtext

\subsection{Description}

A Brainfuck program operates on a linear arrangement of memory cells, each of which can hold a number between 0 and 255.

The following commands are defined:

\begin{tabular}{r|l}
    Symbol & Effect\\
    \hline
    \texttt{>} & Move the memory pointer to the right\\
    \texttt{<} & Move the memory pointer to the left\\
    \texttt{+} & Increase the current memory cell's value\\
    \texttt{-} & Decrease the current memory cell's value\\
    \texttt{.} & Output the current memory cell's value as an ASCII character\\
    \texttt{,} & Read an ASCII character from the user, write it to the current memory cell\\
    \texttt{[} & If the current cell contains a 0, skip to the matching closing bracket\\
    \texttt{]} & If the current cell does not contain a 0, return to the matching opening bracket\\
\end{tabular}

Using the brackets, the programmer can realize a \texttt{while}-loop: The expression

\begin{quotation}
    \texttt{[}<code>\texttt{]}
\end{quotation}

will execute <code> until the current cell is empty.

\subsection{Example}

\lstinputlisting[frame=tbrl,basicstyle=\ttfamily\footnotesize]{../langs/brainfuck/readprint.brainfuck}

\blindtext

\subsection{Variants}

\blindtext
