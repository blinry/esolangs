Befunge is a self-modifying, stack-based, multi-dimensional language. It is similar to Brainfuck, but it does not operate on a tape, but instead on a two-dimensional matrix. The programs have a two-dimensional format, as well.

\subsection{Origin}

Befunge was invented in 1993 by Chris Pressey, and is believed to be the first two-dimensional esoteric programming language. Pressey's motivation was to create a language which was as difficult to compile as possible. Not only the multi-dimensionality complicate the compilation process, but also Befunge's ability to modify its own source code. Befunge's stack-based computational model was inspired by Forth, a programming language relying heavily on a data stack and on reverse Polish notation. According to Pressey \cite{pressey_coleman}, the name “Befunge” was originally a mistyping of “before”, typed by Curtis Coleman in a BBS chat system at 4AM.

\subsection{Description}

Programs in Befunge are a grid of \ascii{} characters with a width of 80 and a height of 25 (a common size of old terminal emulators like the VT100), called the \emph{playfield}. Like in Brainfuck, there is an instruction pointer, which starts at the top-left entry, moving right. When the pointer leaves the playfield, it wraps around to the other side. Encountered instructions are executed. The pointer keeps moving in the same direction, if it is not changed.

Befunge also maintains a stack of values, which can be manipulated in various ways. Befunge has a large instruction set of 36 different commands, but in contrast to other languages, they are really easy to understand:

\begin{description}[labelsep=1em]
    \item[\texttt{+} \texttt{-} \texttt{*} \texttt{/} \texttt{\%}] Pop two values, add/subtract/multiply/integer-divide/modulo them, and push the result back onto the stack.
    \item[\texttt{!}] Pop a value, if it is zero, push 1, otherwise, push 0.
    \item[\texttt{`}] Pop two values, if the first one is smaller, push 1, otherwise 0.
    \item[\texttt{>} \texttt{<} \texttt{\textasciicircum} \texttt{v}] Move the pointer to the right/left/up/down.
    \item[\texttt{?}] Move the pointer in a random direction.
    \item[\texttt{\_}] Pop a value if it is 0, move right, otherwise left.
    \item[\texttt{|}] Pop a value it it is 0, move down, otherwise up.
    \item[\texttt{"}] Turn on stringmode: Until the next “\texttt{"}”, the values of the \ascii{} characters will be pushed.
    \item[\texttt{:}] Duplicate the top stack value.
    \item[\texttt{\textbackslash}] Swap the top two stack values.
    \item[\texttt{\$}] Discard the top stack element.
    \item[\texttt{.}] Print the value of the topmost element and discard it.
    \item[\texttt{,}] Print the \ascii{} character of the topmost element and discard it.
    \item[\texttt{\#}] Bridge: Jump over the next command.
    \item[\texttt{g}] Pop $y$ and $x$ and push the \ascii{} value of the character at that position in the playfield.
    \item[\texttt{p}] Pop $y$, $x$ and $v$, and write the \ascii{} value $v$ to the position $(x,y)$ in the playfield.
    \item[\texttt{\&}] Read an integer from standard input and push it onto the stack.
    \item[\texttt{\~}] Read a character from standard input and push it onto the stack
    \item[\texttt{@}] End the program.
    \item[\texttt{0-9}] Push the respective number onto the stack.
\end{description}

The most important instructions are the direction-changing instructions. For example, this program represents an endless loop:

\begin{lstlisting}
>v
^<
\end{lstlisting}

Note that as calculations are done on the stack, they must be written down in postfix notation. Values larger than 9 must be calculated beforehand, as well, because in the program, only single digits can be specified. For example, this program puts $42\cdot 23$ on the stack:

\begin{lstlisting}
67*83*1-*
\end{lstlisting}

\subsection{Example}

The following program reads characters from the standard input, and performs the ROT13 operation on them, which moves letters by 13 places forward in the alphabet, wrapping from Z back to A. Lowercase and uppercase letters are moved separately. Applying ROT13 to a text twice will restore the original. This operation is sometimes used in forums to hide spoilers from plain view, while making it trivial to en- and decode it.

\lstinputlisting[frame=tbrl,basicstyle=\ttfamily\footnotesize]{befunge/rot13.bf}

\begin{io}
Input: Hello, world!
Output: Uryyb, jbeyq!
\end{io}

To make the paths inside of the program more visible, we show another version here, which has additional (redundant) direction arrows:

\lstinputlisting[frame=tbrl,basicstyle=\ttfamily\footnotesize]{befunge/rot13-paths.bf}

Here is a general idea of how this program works: For each character, it is first determined whether it is in the range a-z or A-Z, and after that, if it is between a-m or A-M, respectively. If it is, its value is increased by 13, otherwise, 13 is subtracted. If the character is no letter at all, it is not changed.

The execution starts with the first character in the first line. When first encountering the \texttt{>}, it is ignored, as the instruction pointer is moving to the right anyway. The \texttt{\~} instruction reads in a character and pushes its \ascii{} value on the stack. This value is then duplicated (\texttt{:}). The first \texttt{"} instruction turns the stringmode on, thus the following character, \texttt{`}, is pushed onto the stack as the value 96 (this is the character directly in front of “a”). The second \texttt{"} closes the stringmode again. After that, the \texttt{`} compares the two topmost stack values, removes them, and then pushes a 1 if the top one was smaller, and a 0 otherwise. Then, the \texttt{!} inverts this “truth value”.

The last instruction on that line, the \texttt{v}, moves the instruction pointer to the first conditional statement in line 2. If there is a 0 on the stack, which is the case if the original letter came after \texttt{`} in the \ascii{} table, the pointer moves right, otherwise left. If it moves right, the original value is duplicated again and now compared to \texttt{z} in exactly the same way as in line 1. This time, the truth value is not inverted, so that we move left if our letter was larger than \texttt{z}, and right otherwise. If moving right, we know that we have a lowercase letter.

In line 3, it is finally compared to \texttt{m}, and control is transferred to the small C-shaped construct in the middle of the program, coming from the right in line 5.

Depending on whether our letter is in the range \texttt{a}--\texttt{m} or in the range \texttt{n}--\texttt{z}, the pointer moves down or up at the \texttt{|} instruction. If it ends up in the lower half, 13 is added, by pushing first a 9 on the stack, then adding it, then doing the same with a 4. Otherwise, it is subtracted. Finally, control is transferred to the first \emph{column} (the detailed version uses bridges (\texttt{\#}) to jump over the \texttt{v} lane), where the resulting letter is finally printed with the \texttt{,} instruction. After that, the first \texttt{>} in line 1 closes the loop, and the process is started again.

Note that when we find that our letter is smaller than \texttt{`} in line 2, we move left, and then down to line 9. Here the same process is repeated for uppercase letters, which are larger than \texttt{@}, smaller or equal to \texttt{Z}, and also dividable into two groups with the \texttt{M}. The inner part is reused. At all other failing tests, control is transferred back to column 1, and the character is eventually output, whether it was changed (in case of letters) or not (in all other cases).

\subsection{Computational Class}

Because of its hard space limitations ($80\cdot 25$ bytes), Befunge is not Turing-complete. Without this limitation, it would be relatively easy to implement a Brainfuck interpreter in it, which would prove Turing-completeness.

\subsection{Implementations}

Pressey's original implementation, bef2c \cite{pressey1993befunge}, compiles Befunge-93 to C, but the resulting program works like an interpreter, which does not allow for much optimization. Some more advanced compilers, like befunjit \cite{toncean_befunge} follow a just-in-time approach: Instead of executing the instructions one by one, they look for the longest static path (which does not contain any conditional execution), and precompiles it, which improves the runtime significantly. If the \texttt{p} instruction changes some of these paths by writing to them, the paths are invalidated and recompiled. Befungee \cite{mcenroe_befunge} is an interpreter written in Python that provides with a graphical debugger. Recently, \emph{Befunge for Android} was released \cite{alexander_befunge}, which provides a “graphical” interface and a step-by-step functionality.

\subsection{Variants}

In 1998, Pressey released \textbf{Befunge-98}, which removes the size restriction of the playfield, making the language Turing-complete. Actually, Befunge-98 is a member of a generalized family of languages, the \textbf{Funge-98} family. Its members are \textbf{Unefunge}, \textbf{Befunge}, and \textbf{Trefunge}, which operate in one, two, or three dimensions respectively. In Trefunge, the \emph{form feed} character (with \ascii{} value 12, formerly used to eject a page from printers), is used to increase the z-coordinate of the source code, thus going to the next “layer” of the program.

There are also many more languages inspired by Befunge: \textbf{Wierd}, created as a collaboration on the Befunge mailing list by Pressey, Ben Olmstead (who would later create Malbolge), and John Colagioia, attempts to trim down the number of instructions. In fact, there is only one instruction: All non-whitespace characters are treated the same. The semantic is defined by the angles formed by lines of characters in the 2D space. \textbf{PATH} is a crossover between Befunge and Brainfuck: The program's source code is one-dimensional, but the data pointer lives in a two-dimensional field.

\subsection{Significance}

In 1996, Pressey invited interested persons to join the \emph{Befunge Mailing List}, where they discussed the language and shared code and ideas \cite{pressey1996welcome}. It later evolved into the \emph{Esoteric Topics Mailing List}, which (along with his \emph{Esoteric Topics} page, see \cref{introduction}) seems to have contributed to the usage of the term “esoteric” in terms of programming languages.

Befunge is still quite popular. There are some actively maintained interpreters/compilers (including Pressey's original implementation), and new variants get published every once in a while. One of the most complex Befunge programs is able to connect to \emph{Internet Relay Chat} servers and perform various tasks there; it consists of over 10,000 visible characters \cite{kallasjoki2008fungot}.
