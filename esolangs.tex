\documentclass{sig-alternate}
\usepackage[T1]{fontenc}
\usepackage[utf8]{inputenc}

\usepackage{blindtext}
\usepackage{microtype}
\usepackage{filecontents}

\title{Esoteric Programming Languages}
\subtitle{A journey to Brainfuck, Befunge, INTERCAL and Malbolge}
\numberofauthors{1}
\author{\alignauthor Sebastian Morr\\\email{sebastian@morr.cc}}

\makeatletter
\def\@copyrightspace{\relax}
\makeatother

\begin{document}
\maketitle

\begin{abstract}
    This paper describes four common esoteric languages, which all have formed their own language category.
\end{abstract}

\section{Introduction}

max 1 column

motivation

general approach/outline

\section{Brainfuck}

One of the most notable esoteric programming language, 

\subsection{History}

\blindtext[1]

\subsection{Description}

\blindtext[1]

\subsection{Example}

\blindtext[1]

\subsection{Variants}

\blindtext[1]

\section{\textsc{INTERCAL}}


\section{Befunge}


\section{Malbolge}

\section{Evaluation}

report experience


comparizon + categorization

\section{Discussion}

interpret results

advantages / disadvantages

\section{Conclusion}

\blindtext[2]

\cite{why05}
\cite{google13}

\bibliographystyle{abbrv}
\bibliography{references}

\end{document}
