\documentclass{sig-alternate}
\usepackage[T1]{fontenc}
\usepackage[utf8]{inputenc}
\usepackage{lmodern}%lighttt

\usepackage{enumitem}
\usepackage{blindtext}
\usepackage{microtype}
    \DisableLigatures[<,>]{family=tt*}
\usepackage{filecontents}
\usepackage{listings}
    \lstset{frame=tbrl,
        numbers=left,
        numberstyle=\tiny,
        basicstyle=\ttfamily\small
        %aboveskip=10pt,
        %belowskip=10pt,
        %literate={`}{\`}1,
    }
    \lstnewenvironment{io}{\lstset{
        numbers=none,
        frameround=tttt,
        basicstyle=\sffamily,
        columns=fullflexible,
        escapechar=!
    }}{}
\usepackage{tabularx}
\usepackage{xcolor}
    \definecolor{medium-blue}{rgb}{0,0,0.5}
\usepackage{hyperref}
    \hypersetup{
        colorlinks, linkcolor={medium-blue},
        citecolor={medium-blue}, urlcolor={medium-blue}
    }
\usepackage[style=numeric,sorting=none]{biblatex}
    \addbibresource{references.bib}
\usepackage[nameinlink]{cleveref}
\usepackage[framemethod=TikZ]{mdframed}

\title{Esoteric Programming Languages}
\subtitle{A weird journey to Brainfuck, Shakespeare, Befunge, INTERCAL and Malbolge}
%\subtitle{\textit{\large We are not really esoteric, it's just that nobody pays much attention to us.}}
\numberofauthors{1}
\author{\alignauthor Sebastian Morr\\\affaddr{Technical University Braunschweig, Germany}\\\email{sebastian@morr.cc}}

\makeatletter
\def\@copyrightspace{\relax}
\makeatother

\begin{document}
\maketitle

\begin{abstract}
    There is this class of programming languages that are not designed to be used for programming. These so-called “esoteric” programming languages have other purposes instead: To entertain, to be beautiful, or to make a point.

    This paper describes and contrasts five stereotypical, widely different esoteric programming languages: Brainfuck, Shakespeare, Befunge, INTERCAL and Malbolge.
\end{abstract}

\section{Introduction}

While programming languages are normally designed to be used productively and being helpful in real-world applications, esoteric programming languages have other goals:

They can be proof-of-concepts, demonstrating how minimal a language syntax can get, while still maintainig an universal character. They might help to prove mathematical theorems or provide bounds in complexity analyses. The design of esoteric programming languages can be seen an art form, and be an expression of human intellect, wit, and asthetic taste. Or they might be created as a kind of competitive sport, a challenge for the language's designer or for its users. Finally, there are joke languages designed to enjoy the authors themselves, the users, or even the readers of the specification.

The term “esoteric” stems from the ancient greek \emph{esoterikos}, meaning “belonging to an inner circle”, and originally referred to Phythagoras' secret teachings. ref It evolved to mean “mystic”, or “having to do with highly theoretical concepts without obvious practical application”. The first usage of the term “esoteric” in the context of “weird programming languages” was probably on a web site called \emph{Esoteric Topics in Computer Programming}, published by Chris Pressey, inventor of Befunge, around 1997 \cite{pressey2005chris}.

For this paper, I picked four well-known esoteric programming languages, that each demonstrate a unique property commonly found in other esoteric languages: \textbf{Brainfuck} attempts to have a \textit{minimal} syntax, which consists only of eight different characters. Nevertheless, it can be shown to be Turing-complete. \textbf{Shakespeare} requires its programs to look like Shakespearean plays, making it a \textit{themed} language. The author(s) of \textbf{INTERCAL} attempted to create a \textit{weird} language, that differed from other languages known at that time in as many aspects as possible. \textbf{Befunge} is a \textit{two-dimensional} lanuguage, the user can use directional commands to control the program flow. Finally, \textbf{Malbolge} was designed to be as hard as possible to use, it took two years to write the first program.

For each language, we are going to look at it's origins, history, and nowadays significance, explain how the language works, give an interesting example and mention some popular variants.

\section{Preliminaries}

We first define some terms which are relevant when discussing esoteric programming languages.

\subsection{Turing machines}

In his 1937 paper \cite{turing1937computable}, Alan Turing defined a simple machine that defines “computing”. It consists of an infinite onedimensional tape with cells that can hold symbols, a head which moves over the tape and can read and write symbols, and a finite state machine, which specifies what to do when a specific symbol is read. This simple architecture is so mighty that it can simulate all other models of computers, and thus is able to compute every computable sequence.

In fact, one construct a Turing machine that reads a description of another Turing machine, plus some input, and then simulates that other machine, this is what we call a \emph{universal Turing machine}.

\subsection{Turing completeness}

Systems are called \emph{Turing complete}, if they can be used to simulate universal Turing machines. These systems are as mighty as the class of Turing machines, meaning they can compute any computable sequence.

Turing completeness is an important property of omnipotent programming languages. It is highly probable that all modern general-purpose programming languages are Turing complete, but not all esoteric programming languages are, which makes it interesting to look at this property.

\subsection{Turing tarpit}

The language is an example of a so-called \emph{Turing tarpit}. This term was coined in 1992 by Alan Perlis, first recipient of the Turing Award, who warned against environments “in which everything is possible but nothing of interest is easy” \cite{perlis1982epigrams}, in reference to geologic asphalt lakes, whose thick consistency slows down movements for everything inside. Turing tarpit languages, like Brainfuck, provide a handful of very general and flexible mechanisms, which can be used to write \emph{any} program, but it is seldomly pratical to do so, because the languages provide so little abstraction that the programs get very long or complicated.

\newpage
\section{Brainfuck}

Brainfuck is a well-known example of a minimalistic programming language, aiming for a small language syntax and small compilers. Its programs consist of only eight characters, nevertheless, it was proven to be Turing-complete.

\subsection{History}

Brainfuck was designed by Urban Müller, a Swiss physics student who in 1992 took over a small online archive for Amiga software \cite{muller1993aminet}. Back than it saw around 40 users per day, today it is the world's largest Amiga archive.

In 1993, Müller uploaded the first Brainfuck compiler on this platform \cite{muller1993240}, in the form of a machine language implementation compiling into an executable of 296 bytes. Along with it came a README with a concise language description and the sentence “Who can program anything useful with it?~:)”, and it included some quite elaborate examples, as well.

As Aminet grew, the compiler became popular amongst the Amiga community, and was later implemented for other platforms. Some thorough sources of information about Brainfuck are ref ref ref.

Brainfuck was proven Turing complete by Daniel Cristofani \cite{cristofani-universal}, by implementing a simple universal Turing machine described by Yurii Rogozhin \cite{rogozhin1996small}.

Nowadays, Brainfuck is probably the best-known esoteric programming language. It is a common programming exercise to implement a Brainfuck compiler in another language (although people also have written Brainfuck compilers in Brainfuck itself). Fans of the language succeeded to implement even smaller compilers than the original version, the smallest one is a MS-DOS binary only 98 bytes large \cite{inte1999entry}.

\subsection{Description}

In it's setup, Brainfuck bears some similarities to Turing machines:

A Brainfuck program operates on an infinite linear arrangement of memory cells, often called \emph{tape}. Each memory cell contains an unsigned byte (a number between 0 and 255), at the beginning of the program, all cells are initialized to 0. Additionally, Brainfuck maintains a pointer to one of the memory cells.

The program source code consists of a sequence of eight symbols, which are defined as follows:

\begin{description}[labelsep=1em]
    \item[\texttt{>}] Move the pointer to the right.
    \item[\texttt{<}] Move the pointer to the left.
    \item[\texttt{+}] Increment the current cell's value by 1.
    \item[\texttt{-}] Decrement the current cell's value by 1.
    \item[\texttt{.}] Output the current cell's value as an ASCII character.
    \item[\texttt{,}] Read an ASCII character from the user, write its value to the current cell.
    \item[\texttt{[}] If the current cell contains a 0, skip to the matching closing bracket.
    \item[\texttt{]}] If the current cell does not contain a 0, return to the matching opening bracket.
\end{description}

Other characters in the source code are ignored (which allows for inline documentation). While in- or decrementing, the cells' values always wrap to stay inside the range of 0 to 255.

Using the brackets, the programmer can realize a \texttt{while}-loop: The expression

\begin{quotation}
    \texttt{[}<code>\texttt{]}
\end{quotation}

will execute <code> until the current cell is 0. For example, the expression

\begin{quotation}
    \texttt{[->+<]}
\end{quotation}

will add the current cell's value to the right neighbouring cell by decrementing the current cell by one, going right, incrementing by one, going left again. This sequence is repeated until the current cell is 0.

Another gadget which we will use in the following example is

\begin{quotation}
    \texttt{+[->+]-}
\end{quotation}

which moves to the right until the current cell's value is 255. The \texttt{+} operators increment the value before each check, so that the loop will terminate when we create a 0 that way. The \texttt{-} operators reset it to it' original value before the cell or the construct is left again.

\subsection{Examples}

The following program reads a sequence of ASCII values from the user, and prints their binary representation. In order to be able to appreciate Brainfuck's unique aesthetics, it is first given in minified form:

\lstinputlisting{brainfuck/ascii-min.b}

\begin{io}
Input: !\texttt{hello}!
Output: !\texttt{0110100001100101011011000110110001101111}!
\end{io}

Let's look at it in more detail. The following commented version, can be broken down into three basic parts: Line 1 sets up the basic memory layout, which is restored for each character the user enters. The programs uses a “sentinel” cell with the special value 255 to facilitate seeking back to the end of the bit array. Line 10 reads the character, lines 11--18 implement a simple shift register to calculate the binary representation. The remaining lines print the binary digits, by calculating the digits' ASCII values (48 or 49), and finally restore the memory layout. Note that lines 21---25 could be replaced with \texttt{++++++++++++++++++++++++++++++++++++++++++++++++}, which would increment the cell's value by 48, but we wanted to demonstrate a more esoteric (and more concise) approach here, which is why the actual code increases the value by 6 eight times.

\lstinputlisting{brainfuck/ascii.b}

\subsection{Variants}

The esolangs wiki \cite{esolang}, a large database of esoteric programming topics, and informal successor of Chris Pressey's Esoteric Topics site, lists 162 articles in the “Brainfuck derivatives” category and 33 “Brainfuck equivalents”, which were all inspired by Müller's original implementation. There are variants which operate on two tapes (\textbf{DoubleFuck}), restrict the cells to binary values, thus making the \texttt{+} and \texttt{-} operations identical (\textbf{Boolfuck}). Some add more operators (like \textbf{Brainfork}, which adds a \texttt{Y} command for forking the process), others try to reduce the command set even further (\textbf{BitChanger} also works on bit cells and defines \texttt{\}} $=$ \texttt{>+}. The effect of the original \texttt{>} can be simulated with \texttt{\}<\}}).
The joke variant \textbf{Ook!} behaves exactly like Brainfuck, but it's operators are pairs of Orangutan words like “Ook. Ook?” for \texttt{>} or “Ook! Ook!” for \texttt{-}.

xxx references

Because of Müller's informal language definition, the various compiler implementations differ in various aspects:

While the general idea assumes an infinitely long tape, actual implementations of course have some kind of memory limit. The original compiler uses a tape of 30,000 cells, with the pointer starting on the leftmost one. Some implementations extend the memory array when the pointer steps out of the allocated range, others crash, others again will wrap around.

The size of one cell was one byte in the original implementation, wrapping around to 255 when subtracting from 0, and many implementations follow that design. Others use 16- or 32-bit numbers, or signed values, allowing negative numbers.

Another implementation difference is about what happens when a Brainfuck programs wants to read a byte, but there is no more input---for example, because the input was a file which has reached the end-of-file condition. In many applications, it is interesting to know that there will be no more input. Müller's implementation leaves the current cell unchanged in this case, others set it to 0, others to -1 (this requires cells which are larger than bytes).

Interestingly, Brainfuck had a much earlier predecessor: In 1964, the theoretical computer scientist Corrado Böhm designed the language $\mathcal{P}''$, to describe a specific family of Turing machines \cite{bohm1964family}. Programs in $\mathcal{P}''$ consist of word over the alphabet $\{R, \lambda, (, )\}$. Böhm showed that this language was Turing-complete long before Brainfuck was implemented.

$\mathcal{P}''$ operates on a left-infinite tape, which can contain symbols of an alphabet $\{a_0, a_1, \dots, a_n\}$. Initially, each memory cell contains $a_0$, the \emph{blank symbol}. The symbols' semantics are as follows:

\begin{description}
    \item[\boldmath$R$] Move the memory pointer to the right.
    \item[\boldmath$\lambda$] Increase the current symbol, then move the memory pointer to the left.
    \item[\boldmath$(q)$] Repeat $q$ while the current symbol does not equal the blank symbol.
\end{description}

Each Brainfuck program can be translated  to a $\mathcal{P}''$ program using the following equivalents:
\begin{align*}
    \text{\texttt{+}} &\rightarrow r = \lambda R &\text{(increase and move left, then move right)}\\
    \text{\texttt{-}} &\rightarrow r' = \overbrace{rrr\dots r}^n &\text{(the symbol will “wrap around”)}\\
    \text{\texttt{>}} &\rightarrow R\\
    \text{\texttt{<}} &\rightarrow r'\lambda &\text{(“decrement” the cell, then $\lambda$)}\\
    \text{\texttt{[}} &\rightarrow (\\
    \text{\texttt{]}} &\rightarrow )
\end{align*}

Note that $\mathcal{P}''$ only has four operators. So called \emph{one instruction set computers} (OISC) go even further, by reducing the number of operations to one. A common example is the “subtract and branch if less than or equal to zero” operation, also known as \texttt{subleq}. It takes three memory pointers $a$, $b$ and $c$ as its operands, subtracts the value at $a$ from the value at $b$, writes the result back to $b$ and jumps to $c$ if it is not positive. One can synthesise all other operators from this single one, for example, an unconditional jump:

\begin{lstlisting}
subleq a a c // jump to c, a-a is always 0
\end{lstlisting}

or addition:

\begin{lstlisting}
subleq c c 2 // c = 0 go to the next line
subleq a c 3 // c' = c-a = -a
subleq c b 4 // b = b-c' = a+b
\end{lstlisting}


\newpage
\section{Shakespeare}

\subsection{Origins}

The Shakespeare Programming Language (SPL, but we will refer to it here as “Shakespeare”) was created by Karl Hasselström and Jon Åslund in 2001 who were studying at the Royal Institute of Technology in Stockholm at the time. According to the documentation \cite{...}, the authors were given an assignment in their university class that involved lexical and syntactical analysis. They knew and liked Brainfuck and Malboge, and so designed and implemented a language that resembled Shakespearian plays.

They didn't implement a compiler, but the compiler output C code instead
%http://www.computerworld.com.au/article/391510/a-z_programming_languages_shakespeare/?pp=2

According to the authors, Shakespeare “combines the expressiveness of BASIC with the user-friendliness of assembly language”.

\subsection{Description}

The text up to the first period is the program's title. It has no meaning.

The next section is a list of all characters in the play, which is Shakespeare's way of declaring variables. A declaration consists of a name (which must be an actual character in one of Shakespeare's plays, followed by a description.

The play is divided into acts and scenes, which are numbered with roman numerals. The numerals act as labels which can be jumped to using goto statements.

There is a group of statements that make the characters enter or leave the stage, that is, \texttt{[Enter/Exit} <character>\texttt{]}.

Characters can talk to each other when they are on the stage.

Any noun represents an integer, either a 1 (if it is a “nice” noun, like “flower” xxx or a neutral one like “chair”), or a -1 (if it isn't, like “politician” xxx). Nouns can be prefixed with adjectives, each multiplying its value by two.

Statements like “You are ...” assign values to the other character on stage. The sentences “Open your heart” and “Speak your mind” are used to output the other character's numerical value, respectively it's correspondent ASCII character. “Listen to your heart” and “Open your mind” can be used to read a number/character to the other character on stage.

There are goto statements like “Let us proceed to act II”, and conditional statements, which consists of a comparative question and a “If so/not, ...”.

Finally, each character has a stack of numbers. A character can push its value onto the stack of another with “Remember me”, and pop it with “Recall ...”.

\subsection{Example}

\lstinputlisting{shakespeare/fibonacci.spl} % TODO remove "use"

\begin{io}
Output:
1 1 2 3 5 8 13 21 34 55 89 144 233 377 610 987 1597 2584
4181 6765 10946 17711 28657 46368 75025 121393 196418
317811 514229 832040 1346269 2178309 3524578
\end{io}

\subsection{Variants}

\blindtext[3]

\subsection{Significance}

%Spread

%http://developers.slashdot.org/story/01/08/31/1126253/the-shakespeare-programming-language

In the late xxx, they were asked by David Touretzky whether they would implement DeCSS in Shakespeare, a software used to decrypt DVDs which was forbidden by xxx. A performance of such a program would be protected by free speech laws and could have benn legally exported to other countries \cite{herrick2011az}. While they didn't do this because of time reasons, this example demonstrates Shakespeare's versality. In fact, in the keynote of ACM's third \emph{History of Programming Languages conference} in 2007, Guy Steele and Richard Gabriel featured a recorded performance of an actual Shakespeare program that outputs factors of two \cite{chapiewski2007computational}.


\newpage
\section{INTERCAL}

\subsection{History}

\blindtext[2]

\subsection{Description}

\blindtext[2]

\subsection{Example}

\lstinputlisting[frame=tbrl,basicstyle=\ttfamily\footnotesize]{../langs/intercal/hello.i}

\blindtext

\subsection{Variants}

\blindtext


\newpage
\section{Befunge}

\subsection{History}

%http://frox25.no-ip.org/~mtve/tmp/bef_maillist_0_520.txt

\blindtext[5]

\subsection{Description}

\blindtext[5]

\subsection{Example}

\lstinputlisting[frame=tbrl,basicstyle=\ttfamily\footnotesize]{../langs/befunge/fibonacci.bf}

\blindtext[5]

\subsection{Variants}

\blindtext[3]


\newpage
\section{Malbolge}

\subsection{History}

\blindtext[2]

\subsection{Description}

\blindtext[2]

\subsection{Example}

\blindtext

\subsection{Variants}

\blindtext


\newpage

\section{Conclusion}

\blindtext[5]

\printbibliography

\end{document}
